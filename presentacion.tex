\documentclass[14pt,xcolor=x11names,compress,aspectratio=169,usenames,dvipsnames]{beamer}%aspectratio=169
\usepackage[spanish,english]{babel}
\usepackage[utf8]{inputenc}
\usepackage{amsmath,amsfonts,amsbsy,amssymb,mathtools,amsthm,bm}
\usepackage{upgreek}
\usepackage{tikz}
\usetikzlibrary{trees,arrows,fadings,mindmap,shapes,decorations.pathreplacing,matrix}
\usepackage{hyperref}
\hypersetup{colorlinks,linkcolor=white,urlcolor=blue}
\usepackage{colortbl}
\usepackage{comment}
\usepackage{subcaption}
\usepackage[T1]{fontenc}
\usepackage{xcolor}
\usepackage{bm}
\setbeamercovered{transparent}
\definecolor{cinves}{rgb}{0.13, 0.7, 0.67}
\usepackage{blkarray}
\usepackage{array}

%\usetheme{Warsaw}
\usetheme[secheader]{Boadilla}
\setbeamertemplate{itemize subitem}[square] 
\useoutertheme[subsection=false,shadow=true]{miniframes}
\useinnertheme{default}
\usefonttheme{serif}

\definecolor{greenlinecolor}{rgb}{0.3,0.7,0.3}
\definecolor{themecolor}{rgb}{0.9,0.9,1}
\definecolor{strongcolor}{rgb}{0.8,1,0.8}
\definecolor{auxiliarcolor}{gray}{0.2}
\definecolor{beamer@blendedblue}{RGB}{0,171,152}
\definecolor{mycolor}{RGB}{0,171,152}
\colorlet{myredblue}{blue!30!red!100}

\setbeamercolor{normal text}{fg=black}\usebeamercolor*{normal text}
\setbeamercovered{dynamic}
\setbeamerfont{frametitle}{size=\Large}
\setbeamercolor{alerted text}{fg=blue}
\setbeamerfont{title like}{shape=\scshape}
\setbeamercolor{mini frame}{fg=white,bg=white}
%\setbeamerfont{frametitle}{shape=\scshape}
\setbeamercolor*{lower separation line head}{bg=white}
\setbeamercolor*{normal text}{fg=black,bg=white}
\setbeamercolor*{eje text}{fg=black}
%\setbeamercolor*{structure}{fg=black}
\setbeamercolor{section in head/foot}{fg=white, bg=mycolor}
\setbeamercolor*{palette tertiary}{fg=blue!90,bg=gray!30}
\setbeamercolor*{palette quaternary}{fg=black,bg=black!10}
\setbeamertemplate{navigation symbols}{}
\beamertemplatenavigationsymbolsempty
\setbeamerfont{page number in head/foot}{size=\tiny}
\setbeamertemplate{footline}[frame number]

\newcommand{\myalert}[1]{\tikz[baseline=(aaa.base)]\node(aaa)[fill=yellow!90,rounded corners=3mm]{#1};}
\newcommand{\myredalert}[1]{\tikz[baseline=(aaa.base)]\node(aaa)[fill=yellow!90,rounded corners=3mm,text=red]{#1};}
\newcommand{\mathalert}[1]{\myalert{\ensuremath{#1}}}

\pdfstringdefDisableCommands{%
    \def\\{}%
    \def\includegraphics[]{}
}

\makeatletter
\setbeamertemplate{title page}{%
    \vbox{}
    \vfill
    \vspace{1cm}% NEW
    \begingroup
    \centering
    \begin{beamercolorbox}[rounded=true,sep=8pt,center]{title}
        \usebeamerfont{title}\inserttitle\par%
        \ifx\insertsubtitle\@empty%
        \else%
        \vskip0.25em%
        {\usebeamerfont{subtitle}\usebeamercolor[fg]{subtitle}\insertsubtitle\par}%
        \fi%    
    \end{beamercolorbox}%
\vskip1em\par
\begin{beamercolorbox}[sep=8pt,center]{author}
\usebeamerfont{author}\insertauthor
\end{beamercolorbox}
\begin{beamercolorbox}[sep=1pt,center]{advisor}
{\small Asesor: José Martínez Bernal}
\end{beamercolorbox}
\begin{beamercolorbox}[sep=8pt,center]{institute}
\usebeamerfont{institute}\insertinstitute \\
Departamento de Matemáticas\\
{\vspace{0.3cm}\includegraphics[height=1.8cm]{cinvesL.jpeg}}
\end{beamercolorbox}
%\vspace{1.5cm}% NEW
\begin{beamercolorbox}[sep=8pt,center]{date}
\usebeamerfont{date}\insertdate
\end{beamercolorbox}\vskip0.5em
%    {\usebeamercolor[fg]{titlegraphic}\inserttitlegraphic\par}
\endgroup
%  \vfill
}
\makeatother

\geometry{papersize={20.1cm,10.95cm}}


\title[CINVESTAV]{\textbf{Algoritmo de Metropolis-Hastings\\ y Teorema de Recurrencia de Polya}}
\author[JesusEnriqueBejaranoOchoa]{\textbf{Jesús Enrique Bejarano Ochoa}}
\institute[]{\small{Centro de Investigación y de Estudios Avanzados del I.P.N., M\'exico}}
\date[15/02/21]{\scriptsize{28 de Febrero de 2021}}
\logo{\includegraphics[height=1.2cm]{cinvesL.jpeg}}

\newcommand{\bD}{\mathbb{D}}
\newcommand{\bC}{\mathbb{C}}
\renewcommand{\Re}{\operatorname{Re}}
\newcommand{\cB}{\mathcal{B}}
\newcommand{\bP}{\mathbb{P}}
\renewcommand{\bar}{\overline}
\newcommand{\cI}{\mathcal{I}}
\newcommand{\Reff}{R_{\mathrm{eff}}}
\newcommand{\Weff}{W_{\mathrm{eff}}}
\newcommand{\La}{\Lambda}
\newcommand{\la}{\lambda}
\newcommand{\vro}{\varrho}
\newcommand{\eps}{\varepsilon}
\newcommand{\nf}{\infty}
\newcommand{\Om}{\Omega}
\newcommand{\om}{\omega}
\newcommand{\si}{\sigma}
\newcommand{\ga}{\gamma}
\newcommand{\Ga}{\Gamma}
\newcommand{\Ph}{\Phi}
\newcommand{\ph}{\varphi}
\newcommand{\tht}{\theta}
\newcommand{\eqdef}{\coloneqq}
\newcommand{\sm}{\setminus}
\renewcommand{\phi}{\varphi}
%\renewcommand{\psi}{\varpsi}
\renewcommand{\sp}{\operatorname{sp\,}}
\newcommand{\bN}{\mathbb{N}}
\newcommand{\bR}{\mathbb{R}}
\newcommand{\bT}{\mathbb{T}}
\newcommand{\bZ}{\mathbb{Z}}
\renewcommand{\Re}{\operatorname{Re}}
\renewcommand{\Im}{\operatorname{Im}}
\DeclareMathOperator{\wind}{wind}
\newtheorem{thm}{Teorema}[section]
\newtheorem{prop}[thm]{Proposición}
\newtheorem{lem}[thm]{Lema}
\newtheorem{cor}[thm]{Corolario}
\newtheorem{defi}[thm]{Definición}
\theoremstyle{remark}
\newtheorem*{remark}{Remark}

\newcommand{\mathcolor}{\textcolor}
\newcommand{\emphdef}{\textbf}


\useoutertheme{infolines}

\setbeamertemplate{headline}[default]
\setbeamertemplate{navigation symbols}{}
\addtobeamertemplate{footline}{
    \leavevmode%
    \hbox{%
    \begin{beamercolorbox}[wd=\paperwidth,ht=2.75ex,dp=.5ex,right,rightskip=1em]
    {mycolor}%
\usebeamercolor[fg]{navigation symbols}\insertslidenavigationsymbol%
%\insertframenavigationsymbol%
%\insertsubsectionnavigationsymbol%
%\insertsectionnavigationsymbol%
%\insertdocnavigationsymbol%
\insertbackfindforwardnavigationsymbol%
    \end{beamercolorbox}%
    }%
    \vskip0.5pt%
}{}
%\setbeamerfont{page number in head/foot}{}
%\setbeamertemplate{footline}[frame number]

\begin{document}

\tikzset{my state/.style={circle,draw, minimum width=1cm}}
\setbeamercovered{invisible}
\maketitle

\begin{frame}{Primera parte: Algoritmo de Metropolis-Hastings}
    \begin{enumerate}
        \item
        Matrices no negativas
        %\begin{enumerate}
        %    \item
        %    Digráfica asociada
        %    \item
        %    Irreducibilidad, periodicidad y regularidad
        %\end{enumerate}
        \bigskip\pause
        \item
        Cadenas de Markov
        %\begin{enumerate}
        %    \item
        %    Clasificación de estados
        %    \item
        %    Distribuciones estacionarias
        %    \item
        %    Teorema de Perron-Frobenius
        %\end{enumerate}
        \bigskip\pause
        \item
        Algoritmo de Metropolis-Hastings
        %\begin{enumerate}
        %    \item
        %    Algoritmo de Metropolis
        %    \item
        %    Algoritmo de Metropolis-Hastings
        %    \item
        %    Ejemplos numéricos
        %\end{enumerate}
    \end{enumerate}
\end{frame}

\begin{frame}{Segunda parte: Teorema de recurrencia de Polya}
    \begin{enumerate}
        \item
        Teorema de recurrencia de Polya
        \bigskip\pause
        \begin{enumerate}
            \item
            {\normalsize Vía redes eléctricas}
            %\begin{enumerate}
            %    \item
            %    Problema de Dirichlet discreto
            %    \item
            %    Principio de Rayleigh
            %    \item
            %    Resistencia efectiva y recurrencia
            %    \item
            %    Bosquejo de la demostración
            %\end{enumerate}
            \bigskip\pause
            \item
            {\normalsize Vía funciones especiales}
            %\begin{enumerate}
            %    \item
            %    Funciones generadoras y la transformada de Borel
            %    \item
            %    Funciones modificadas de Bessel
            %    \item 
            %    Bosquejo de la demostración
            %\end{enumerate}
        \end{enumerate}
    \end{enumerate}
\end{frame}

\begin{frame}{Digráfica asociada a una matriz no negativa}
    \textbf{Correspondencia entre matrices y digráficas} 
    \bigskip\pause
    \begin{itemize}
        \item
        Explicar la teoría a través de digráficas con pesos positivos
        \bigskip\pause

        \item
        Argumentos combinatorios
        \bigskip\pause

        \item
        Mejor visualización
    \end{itemize}
\end{frame}

\begin{frame}{Digráfica asociada a una matriz no negativa}
\textbf{Ejemplos}
\bigskip 

\begin{equation*}
       \begin{bmatrix}
        0 & 1/2 & 0   & 0   & 0 \\
        1 & 0   & 1/2 & 0   & 0 \\
        0 & 1/2 & 0   & 1/2 & 0 \\
        0 & 0   & 1/2 & 0   & 1 \\
        0 & 0   & 0   & 1/2 & 0
    \end{bmatrix}\qquad\longleftrightarrow\qquad
    \begin{tikzpicture}[baseline={([yshift=-.5ex]current bounding box.center)}]
        \node[my state] (1) at (0,0) {$1$};
        \node[my state] (2) at (2,0) {$2$};
        \node[my state] (3) at (4,0) {$3$};
        \node[my state] (4) at (6,0) {$4$};
        \node[my state] (5) at (8,0) {$5$};

        \path[every node]
        (1) edge[->,>=latex,bend right, line width=1pt] node[below] 
        {\mathcolor{blue}{$1$}} (2);
        \path[every node]
        (2) edge[->,>=latex,bend right, line width=1pt] node[above] 
        {\mathcolor{blue}{$\frac{1}{2}$}} (1);
        \path[every node]
        (2) edge[->,>=latex,bend right, line width=1pt] node[below] 
        {\mathcolor{blue}{$\frac{1}{2}$}} (3);
        \path[every node]
        (3) edge[->,>=latex,bend right, line width=1pt] node[above] 
        {\mathcolor{blue}{$\frac{1}{2}$}} (2);
        \path[every node]
        (3) edge[->,>=latex,bend right, line width=1pt] node[below] 
        {\mathcolor{blue}{$\frac{1}{2}$}} (4);
        \path[every node]
        (4) edge[->,>=latex,bend right, line width=1pt] node[above] 
        {\mathcolor{blue}{$\frac{1}{2}$}} (3);
        \path[every node]
        (4) edge[->,>=latex,bend right, line width=1pt] node[below] 
        {\mathcolor{blue}{$\frac{1}{2}$}}(5);
        \path[every node]
        (5) edge[->,>=latex,bend right, line width=1pt] node[above] 
        {\mathcolor{blue}{$1$}} (4);
    \end{tikzpicture}
\end{equation*}

\bigskip\pause

\begin{equation*}
    \begin{bmatrix}
        0&1/4&1/4&1/4&1/4\\
        1/4&0&1/4&1/4&1/4\\
        1/4&1/4&0&1/4&1/4\\
        1/4&1/4&1/4&0&1/4\\
        1/4&1/4&1/4&1/4&0
    \end{bmatrix}\qquad\longleftrightarrow\qquad
    \begin{tikzpicture}[baseline={([yshift=-.5ex]current bounding box.center)}]
        \def \n {5}
        \def \radius {1.5cm}
        \def \margin {4}
        
        \foreach \k in {1,...,\n}
        {
            \node[my state] (\k) at ({360/\n *(\k-1)}:\radius) {$\k$};
        }

        \path[every node]
        (1) edge[<->,>=latex,line width=1pt] node[above right] 
        {\mathcolor{blue}{$\frac{1}{4}$}} (2);
        \path[every node]
        (1) edge[<->,>=latex,line width=1pt] node[below right] 
        {\mathcolor{blue}{$\frac{1}{4}$}} (5);
        \path[every node]
        (1) edge[<->,>=latex,line width=1pt] node[right] {} (4);
        \path[every node]
        (1) edge[<->,>=latex,line width=1pt] node[right] {} (3);
        \path[every node]
        (2) edge[<->,>=latex,line width=1pt] node[right] {} (3);
        \path[every node]
        (2) edge[<->,>=latex,line width=1pt] node[right] {} (4);
        \path[every node]
        (2) edge[<->,>=latex,line width=1pt] node[right] {} (5);
        \path[every node]
        (3) edge[<->,>=latex,line width=1pt] node[right] {} (4);
        \path[every node]
        (3) edge[<->,>=latex,line width=1pt] node[right] {} (5);
        \path[every node]
        (4) edge[<->,>=latex,line width=1pt] node[right] {} (5);
        
    \end{tikzpicture}
\end{equation*}
\end{frame}

\begin{frame}{Matrices irreducibles}
    \textbf{Relación de comunicación}
    \begin{equation*}
    \begin{bmatrix}
        0 & 0 & \ast & 0 \\
        \ast & 0 & 0 & 0 \\
        0 & \ast & 0 & \ast \\
        0 & 0 & 0 & 0 
    \end{bmatrix}
    \qquad \longleftrightarrow \qquad
    \begin{tikzpicture}[baseline={([yshift=-.5ex]current bounding box.center)}]
        \node[my state] (1) at (0,0) {$1$};
        \node[my state] (2) at (2,2) {$2$};
        \node[my state] (3) at (4,2) {$3$};
        \node[my state] (4) at (6,0) {$4$};

        \draw[->,>=latex,line width=1pt] (1) to (2);
        \draw[->,>=latex,line width=1pt] (2) to (3);
        \draw[->,>=latex,line width=1pt] (3) to (1);
        \draw[->,>=latex,line width=1pt] (4) to (3);
    \end{tikzpicture}
    \end{equation*}
    \begin{align*}
        \qquad \qquad\qquad  \qquad \qquad 1 & \leftrightarrow 3\\
        \qquad\qquad  \qquad\qquad  \qquad 1 & \not\leftrightarrow 4
    \end{align*}

\alt<2>{
Una matriz es \emphdef{irreducible} si todos sus estados se comunican.
}
{\center
$i\leftrightarrow j$ si $(A^k)_{ij}>0$ y $(A^\ell)_{ji}>0$ para algunos
$k,\ell$
}

\end{frame}

\begin{frame}{Matrices aperiódicas}
    \textbf{Período del estado $\bm{j}$: } 
    $d_j=\gcd\{n\in\bN\colon\ \exists\text{ ciclo en $j$ de longitud $n$}\}$
\bigskip\pause

\begin{equation*}
        \begin{bmatrix}
        0 & \ast & 0   & 0   & 0 \\
        \ast & 0   & \ast & 0   & 0 \\
        0 & \ast & 0   & \ast & 0 \\
        0 & 0   & \ast & 0   & \ast \\
        0 & 0   & 0   & \ast & 0
        \end{bmatrix}
        \qquad\longleftrightarrow\qquad
    \begin{tikzpicture}[baseline={([yshift=-.5ex]current bounding box.center)}]
        \node[my state] (1) at (0,0) {$1$};
        \node[my state] (2) at (2,0) {$2$};
        \node[my state] (3) at (4,0) {$3$};
        \node[my state] (4) at (6,0) {$4$};
        \node[my state] (5) at (8,0) {$5$};

        \path[every node]
        (1) edge[->,>=latex,line width=1pt,bend right]  (2);
        \path[every node]
        (2) edge[->,>=latex,line width=1pt,bend right]  (1);
        \path[every node]
        (2) edge[->,>=latex,line width=1pt,bend right]  (3);
        \path[every node]
        (3) edge[->,>=latex,line width=1pt,bend right]  (2);
        \path[every node]
        (3) edge[->,>=latex,line width=1pt,bend right]  (4);
        \path[every node]
        (4) edge[->,>=latex,line width=1pt,bend right]  (3);
        \path[every node]
        (4) edge[->,>=latex,line width=1pt,bend right]  (5);
        \path[every node]
        (5) edge[->,>=latex,line width=1pt,bend right]  (4);
    \end{tikzpicture},
    \qquad \bm{d=2}
\end{equation*}

\bigskip\pause

\begin{equation*}\label{eq:matriz_aperiodica}
    \begin{bmatrix}
        0  & 0  & \ast  & 0  \\
        \ast  & 0  & 0  & 0  \\
        0  & \ast& 0  & \ast  \\
        0  & \ast& 0  & 0  
    \end{bmatrix}
    \qquad\longleftrightarrow\qquad
    \begin{tikzpicture}[baseline={([yshift=-.5ex]current bounding box.center)}]
        \node[my state] (1) at (0,2) {$1$};
        \node[my state] (2) at (2,2) {$2$};
        \node[my state] (3) at (1,0) {$3$};
        \node[my state] (4) at (3,0) {$4$};

        \path[every node]
        (1) edge[->,>=latex,line width=1pt]   (2);
        \path[every node]
        (2) edge[->,>=latex,line width=1pt]   (3);
        \path[every node]
        (2) edge[->,>=latex,line width=1pt]   (4);
        \path[every node]
        (3) edge[->,>=latex,line width=1pt]   (1);
        \path[every node]
        (4) edge[->,>=latex,line width=1pt]   (3);
    \end{tikzpicture},
    \qquad \bm{d=1}
\end{equation*}
\end{frame}

\begin{frame}{Matrices regulares}
Una matriz no negativa $A$ es \emphdef{regular} si
    \alt<2->
    {
    existen caminos de longitud $k$ entre cualquiera dos vértices
    para alguna $k$
    }
    {
     $A^k>0$ para alguna $k$.
    }

    \bigskip\pause

    Tenemos las siguientes afirmaciones
    \begin{itemize}
        \item
        Si $A$ es regular, entonces $A$ es irreducible.

        \item
        Si $A$ es regular, entonces $A$ es aperiódica.
        ($A^k>0$ y también $A^{k+1}>0$...)

        \item
        Si $A$ es regular, entonces $A$ es irreducible y aperiódica.
    \end{itemize}
    \bigskip\pause

    \begin{thm}
        $A$ es regular si y sólo si $A$ es irreducible y aperiódica.
    \end{thm}
\end{frame}

\begin{frame}{Cadenas de Markov}
Un vector $\pi\in\bR^N$ es \emphdef{ditribución de probabilidad} si 
$\pi\geq 0$ y $\sum_i\pi_i=1$.

\bigskip

Una matriz $P$ es \emphdef{estocástica} si $P\geq 0$ y cada columna de $P$ es una
distribución.

\bigskip
\bigskip
\pause

\emphdef{Cadena de Markov:} $(x_0, Px_0, P^2x_0, \dots)$ donde $P$ se llama
la \emphdef{matriz de transición} y $x_0$ la \emphdef{distribución inicial}
\begin{itemize}
    \item
    $(P^nx_0)_n$
    
    \item
    $(P, x_0)$

    \item
    $P$
\end{itemize}

\bigskip
\pause

\begin{block}{Objetivo}
    Analizar el comportamiento asintótico de la cadena de Markov
\end{block}


\end{frame}

    
%\begin{frame}{Clasificación de estados}
%El estado $i$ es \emphdef{recurrente} si
%\[
%    \sum_{k=0}^\infty (P^k)_{ii}=\infty
%\]
%y transitorio en caso contrario.
%
%\bigskip
%\pause
%
%\begin{thm}
%    Si $i$ es recurrente y $i\leftrightarrow j$ entonces $j$ también lo es.
%\end{thm}
%
%\bigskip
%\pause
%
%\begin{thm}
%    Siempre existe al menos un estado recurrente.
%\end{thm}
%
%\end{frame}

\begin{frame}{Distribuciones estacionarias}
La distribución $\pi$ es \emphdef{estacionaria} si $P\pi=\pi$.

\bigskip
\bigskip
\pause

\begin{thm}
    Sea $P$ una matriz estocástica. Entonces las siguientes condiciones
    son equivalentes.
    \begin{enumerate}[(i)]
        \item
        $P$ es regular

        \item
        $P$ es irreducible y existe una única distribución $\pi>0$ tal
        que $P^nx\to\pi$ cuando $n\to\infty$ para cualquier distribución 
        $x$
    \end{enumerate}
\end{thm}
    $P$ es una aplicación contractiva en el espacio de todas las distribuciones
    de probabilidad...

\end{frame}

\begin{frame}{Distribuciones reversibles}
    La distribución $\pi$ es \emphdef{reversible} para $P$ si
    \[
        P_{ij}\pi_j = P_{ji}\pi_i
    \]

    \bigskip
    \pause

    \begin{thm}
        Si $\pi$ es reversible para $P$, entonces $\pi$ es 
        estacionara para $P$.
    \end{thm}
    
    \begin{align*}
        (P\pi)_i&= \sum_{j=1}^N P_{ij}\pi_j\\
        &=\Big(\sum_{j=1}^N P_{ji}\Big)\pi_i\\
        &= \pi_i
    \end{align*}
\end{frame}

%\begin{frame}{Teorema de Perron-Frobenius}
%\begin{thm}
%    Si $A\geq 0$ es irreducible entonces se cumple lo siguiente.
%    \begin{enumerate}
%        \item 
%        El radio espectral $r$ de $A$ es positivo y es valor propio de $A$
%        
%        \item
%        La multiplicidad algebráica de $r$ es igual a $1$
%        
%        \item
%        Existe un único vector propio $v>0$ de $A$ asociado a $r$, es decir, satisface
%        $Av=rv$. Además, el vector $v$ es una distribución de probabilidad.
%    \end{enumerate}
%\end{thm}
%\end{frame}

\begin{frame}{Cadena de Metropolis-Hastings}
    Hemos visto que ciertas matrices $P$ tienen una distribución estacionaria
    $\pi$. ¿Se cumple el recíproco? \mathcolor{red}{\bf Sí}

    \bigskip
    \bigskip\pause

    Necesitamos
    \begin{enumerate}
        \item
        La distribución $\pi$

        \item
        Una matriz regular $Q$ llamada la matriz \emphdef{propuesta}
    \end{enumerate}

    \bigskip

    Para producir la \emphdef{cadena de MH} $M$. De manera que $\pi$ es reversible
    para $M$.

\bigskip

Si la cadena propuesta $Q$ es reversible respecto a $\pi$,
\alt<3->{
\mathcolor{red}{por lo general, esto no sucede: }
\[
    Q_{ij}\pi_j\mathcolor{red}{\ \bm >\ } Q_{ji}\pi_i
\]
MH es un {\it ajuste} a $Q$ para obtener la reversibilidad.
}{es decir,
\[
    Q_{ij}\pi_j = Q_{ji}\pi_i
\]
entonces $M=Q$
}
%    \bigskip\pause
%
%    La \emphdef{cadena de Metropolis-Hastings} se define por
%    \[
%        M_{ij}=\frac{1}{\pi_j}\min\{\pi_iQ_{ji},\pi_jQ_{ij}\}
%    \]
%    y ponemos
%    \[
%        M_{jj}=1-\sum_{j\neq i}M_{ij}
%    \]
%
%    \bigskip\pause
%
%    de manera que $\pi$ es reversible para $M$
%    \[
%        M_{ij}\pi_j = M_{ji}\pi_i
%    \]
\end{frame}

%\begin{frame}{Caminatas aleatorias}
%Dos enfoques:
%
%\bigskip
%\bigskip
%
%\emphdef{Proceso estocástico} en tiempo discreto de variables aleatorias simples
%$\xi_1,\xi_2,\dots$
%\[
%    \xi_j\colon V\rightarrow V
%\]
%que satisfacen una condición de Markov de acuerdo a una matriz de transición $P$.
%
%\bigskip
%\bigskip
%\bigskip
%
%\emphdef{Sucesión de estados} donde el $n$-ésimo estado se toma con probabilidad
%$P_{\ast n}$, $P$ es la matriz de transición.
%\end{frame}

\begin{frame}{Cadena de Metropolis-Hastings}
La \emphdef{cadena de Metropolis-Hastings} se define por
\begin{align*}
    M_{ij}&=Q_{ij}\min\Big\{\frac{\pi_iQ_{ji}}{\pi_jQ_{ij}},1\Big\}\\
    M_{jj}&=1-\sum_{i\neq j}M_{ij}
\end{align*}

\bigskip\pause

$\pi$ es reversible para $M$
\[
    M_{ij}\pi_j=\min\{\pi_iQ_{ji},\pi_jQ_{ij}\}=M_{ji}\pi_i
\]


\end{frame}


\begin{frame}{Algoritmo de Metropolis-Hastings}
%\begin{thm}
%    Supongamos que la matriz propuesta $Q$ es irreducible, aperiódica y satisface
%    $Q_{ij}>0$ si y sólo si $Q_{ji}>0$. Si además la distribución $\pi>0$, entonces
%    la cadena de Metropolis-Hastings es una cadena de Markov irreducible y aperciódica
%    que además es reversible respecto a la distribución $\pi$.
%\end{thm}
Una \emphdef{caminata aleatoria} en una gráfica es un proceso que inicia
en un vértice y avanza en cada paso a un vértice vecino de manera aleatoria.

\bigskip
    \emphdef{Algoritmo de MH}


    \begin{enumerate}
        \item
        La caminata está ubicada en el estado $j$ en este momento.
        Escogemos a $i$ de acuerdo a la matriz propuesta, es
        decir, con probabilidad $Q_{\ast j}$
        
        \item
        Si $\pi_j Q_{ij}\leq \pi_i Q_{ji}$, entonces la siguiente posición de la
        caminata será $i$
        
        \item
        Si $\pi_j Q_{ij}>\pi_i Q_{ji}$, lanzamos una moneda de Bernoulli con probabilidad $\pi_i
        Q_{ji}/\pi_j Q_{ij}$ para escoger al estado $i$ y probabilidad $1-\pi_i
        Q_{ji}/\pi_j Q_{ij}$ de escoger al estado $j$
    \end{enumerate}

\bigskip\pause

{\it La caminata favorece incrementos en $\pi$ con probabilidad de $i\rightarrow j$
alta y  $j\rightarrow i$ baja.}
\end{frame}

\begin{frame}{Puntos importantes de MH}
\begin{itemize}
    \item
    Necesitamos una cadena propuesta $Q$. El performance del algoritmo depende de $Q$.

    \bigskip

    \item
    Cuando el estado candidato es rechazado, el estado actual permanece.

    \bigskip

    \item
    El algoritmo no depende de la constante de normalización de $\pi$. Podríamos
    reemplazarla por $C\cdot\pi$.

    \bigskip

    \item
    Cuando la cadena propuesta es simétrica, el algoritmo sólo depende del cociente
    $\pi_i/\pi_j$
\end{itemize}

\end{frame}

%\begin{frame}{Algoritmo de Metropolis}
%Cuando la matriz propuesta es simétrica, estamos hablando del 
%\emphdef{algoritmo de Metropolis}. \pause En este caso, la \emphdef{cadena de Metropolis}
%está dada por
%\[
%    M_{ij} = Q_{ij} \min\Big\{\frac{\pi_i}{\pi_j},1\Big\} 
%\]
%\end{frame}

\begin{frame}{Algoritmo de Metropolis}
Cuando la matriz propuesta es simétrica, estamos hablando del 
\emphdef{algoritmo de Metropolis}.
\bigskip

    \begin{enumerate}
        \item
        La caminata está ubicada en el estado $j$ en este momento.
        Escogemos a $i$ de acuerdo a la matriz propuesta, es
        decir, con probabilidad $Q_{\ast j}$
        
        \item
        Si $\pi_j\leq \pi_i$, entonces la siguiente posición de la
        caminata será $i$
        
        \item
        Si $\pi_j>\pi_i$, lanzamos una moneda de Bernoulli con probabilidad
        $\pi_i/\pi_j$ para escoger al estado $i$ y probabilidad
        $1-\pi_i/\pi_j$ de escoger al estado $j$
    \end{enumerate}

\bigskip\pause

{\it Si $\pi_i<\pi_j$ pero $\pi_i$ está muy cercano a $\pi_j$, es muy posible que
nos quedemos con $i$. El algoritmo no favorece saltos muy pronunciados.}
\end{frame}

%\begin{frame}{Ejemplo del alg. de Metropolis}
%Si incrementa $\pi$... Si no, necesitamos que $\pi_i$ esté muy cerca de $\pi_j$...
%
%\begin{equation*}
%    \begin{tikzpicture}[scale=.75, baseline={([yshift=-.5ex]current bounding box.center)}]
%        \node[my state] (1) at (0,2) {$1$};
%        \node[my state] (2) at (2,2) {$2$};
%        \node[my state] (3) at (0,0) {$3$};
%        \node[my state] (4) at (0,4) {$4$};
%
%        \path[every node]
%        (1) edge[->,bend left=20] node[above] {$6/10$} (2);
%        \path[every node]
%        (2) edge[->,bend left=20] node[above] {} (1);
%        \path[every node]
%        (2) edge[->,bend right=50] node[above] {} (4);
%        \path[every node]
%        (2) edge[->,bend left=40] node[above] {} (3);
%        \path[every node]
%        (1) edge[->,bend right=20] node[left] {$1/10$} (3);
%        \path[every node]
%        (3) edge[->,bend right=20] node[above] {} (1);
%        \path[every node]
%        (3) edge[->,bend right=60] node[above] {} (2);
%        \draw[->]
%        (3) to[bend left=120, distance=2.5cm] (4);
%        \path[every node]
%        (1) edge[->,bend left=20] node[left] {$3/10$} (4);
%        \path[every node]
%        (4) edge[->,bend left=20] node[above] {} (1);
%        \path[every node]
%        (4) edge[->,bend left=70] node[above] {} (2);
%        \path[every node]
%        (4) edge[->,bend right=100] node[above] {} (3);
%    \end{tikzpicture},\qquad\quad
%    \pi = 
%    \begin{bmatrix}
%    2\\ 3\\ 4\\ 1
%    \end{bmatrix}
%\end{equation*}
%
%\alt<2>{
%Comenzando en $2$...
%
%$2\xrightarrow{6/10} 1$, como $\pi_2=3>\pi_1=2$,
%\begin{align*}
%    \pi_4/\pi_1 &= 2/3\\
%    \text{lanzamos moneda }\underbrace{2/3}_{2}&,\quad \underbrace{1/3}_{1}\\
%\end{align*}
%}
%{
%Comenzando en $1$...
%
%$1\xrightarrow{6/10} 2$, $2$ es aceptado porque $\pi_1\leq \pi_2$
%
%$1\xrightarrow{3/10} 4$, como $\pi_1>\pi_4$, 
%\begin{align*}
%    \pi_4/\pi_1 &= 1/2\\
%    \text{lanzamos moneda }\underbrace{1/2}_{1}&,\quad \underbrace{1-1/2}_{4}\\
%\end{align*}
%}
%
%\end{frame}

\begin{frame}{Ejemplo del alg. de MH}
\begin{equation*}
    \begin{tikzpicture}[scale=.75, baseline={([yshift=-.5ex]current bounding box.center)}]
        \node[my state] (1) at (0,2) {$1$};
        \node[my state] (2) at (3,2) {$2$};
        \node[my state] (3) at (0,0) {$3$};
        \node[my state] (4) at (0,4) {$4$};

        \path[every node]
        (1) edge[->,>=latex,line width=1pt,bend left=20] node[above] 
        {\small \mathcolor{red}{$60/100$}} (2);
        \path[every node]
        (2) edge[->,>=latex,line width=1pt,bend left=20] node[below] 
        {\small \mathcolor{red}{$99/100$}} (1);
        \path[every node]
        (2) edge[->,>=latex,line width=1pt,bend right=50] node[above] {} (4);
        \path[every node]
        (2) edge[->,>=latex,line width=1pt,bend left=40] node[above] {} (3);
        \path[every node]
        (1) edge[->,>=latex,line width=1pt,bend right=20] node[left] {} (3);
        \path[every node]
        (3) edge[->,>=latex,line width=1pt,bend right=20] node[above] {} (1);
        \path[every node]
        (3) edge[->,>=latex,line width=1pt,bend right=60] node[above] {} (2);
        \draw[->,>=latex,line width=1pt]
        (3) to[bend left=120, distance=2.5cm] (4);
        \path[every node]
        (1) edge[->,>=latex,line width=1pt,bend left=20] node[left] {} (4);
        \path[every node]
        (4) edge[->,>=latex,line width=1pt,bend left=20] node[above] {} (1);
        \path[every node]
        (4) edge[->,>=latex,line width=1pt,bend left=70] node[above] {} (2);
        \path[every node]
        (4) edge[->,>=latex,line width=1pt,bend right=100] node[above] {} (3);
    \end{tikzpicture},\qquad\quad
    \pi = 
    \begin{bmatrix}
    2\\ 3\\ 4\\ 1
    \end{bmatrix}
\end{equation*}

\alt<2>
{
$1\xrightarrow{\mathcolor{red}{60/100}} 2$, como $Q_{12}\pi_1> Q_{21}\pi_2$, 
\begin{align*}
    \text{lanzamos moneda }\underbrace{10/11}_{2},\quad \underbrace{1/11}_{1}
\end{align*}
}
{
$2\xrightarrow{\mathcolor{red}{99/100}} 1$ 
es aceptado porque $\pi_2Q_{21}\leq \pi_1Q_{12}$
}
\end{frame}


%\begin{frame}{Elección de $Q$}
%Se sabe lo suficiente para el estudio práctico de simulaciones.
%Tenemos familias de cadenas propuestas $Q$.
%
%\bigskip
%
%\begin{itemize}
%    \item
%    Dada una distribución $q_1$, ponemos
%    \[
%        Q(i,j) = q_1(|i-j|)
%    \]
%
%    \item
%    Dada una distribución $q_2$, ponemos
%    \[
%        Q(i,j)=q_2(i)
%    \]
%    
%
%    \item
%    Si $\pi\varpropto\psi h$, donde $h$ es una distribución y $\psi$ es
%    uniformemente acotada, ponemos
%    \[
%        Q(i,j)=h(i)
%    \]
%\end{itemize}
%\end{frame}


%\begin{frame}{Alcance de $Q$}
%\emphdef{Alcance} es la diferencia entre el valor más alto y el más bajo de la 
%distribución.
%
%\bigskip
%
%Los valores de $Q$ influyen en el comportamiento de la cadena
%\begin{itemize}
%    \item
%    tasa de aceptación
%
%    \item
%    cobertura de estados (o vértices)
%\end{itemize}
%
%\bigskip
%\bigskip
%
%Si el alcance es muy grande... Los estados candidatos estarán muy lejos del 
%estado actual y tendrán poca probabilidad de ser aceptados.
%
%\bigskip
%
%Reducir el alcance corrige este problema. Si el alcance es muy pequeño...
%La cadena tomará mucho tiempo en recorrer la cadena por completo. Estados
%con poca probabilidad no serán visitados de manera correcta (undersampled).
%
%\end{frame}

%\begin{frame}{Caminatas recurrentes y transitorias}
%Una caminata aleatoria que inicia en $0$ es \emphdef{recurrente} si la probabilidad
%de que regrese a $0$ es igual a $1$. Si esa probabilidad es $p<1$,
%entonces es \emphdef{transitoria}
%
%%\bigskip
%%\bigskip
%%
%%Equivalentemente, es recurrente si la probabilidad de regresar una {\it infinidad}
%%de veces a $0$ es igual a $1$.
%%
%%\bigskip
%%\bigskip
%%
%%Es transitoria si la probabilidad de regresar una {\it cantidad finita} de veces
%%a $0$ es igual a $1$.
%    
%
%\end{frame}

\begin{frame}{Resumen de la primera parte}
\begin{itemize}
    \item
    Una matriz $A\geq 0$ es regular si y sólo si $A$ es irreducible y
    aperiódica.

    \bigskip
    \bigskip

    \item
    Una cadena regular tiene una única distribución $\pi>0$ estacionaria.

    \bigskip
    \bigskip
    
    \item
    Dada $\pi$ y una cadena propuesta $Q$, el algoritmo de MH permite
    aproximar $\pi$
    \begin{itemize}
        \item
        {\normalsize
        A través de la sucesión $Mx_0,M^2x_0,M^3x_0,\dots$
        }

        \bigskip

        \item
        {\normalsize
        A través de una {\it caminata aleatoria} en la digráfica de $Q$
        }
    \end{itemize}
\end{itemize}
\end{frame}

\begin{frame}{Teorema de recurrencia de Polya}
Una caminata aleatoria que inicia en $0$ es \emphdef{recurrente} si la probabilidad
de que regrese a $0$ es igual a $1$. Si esa probabilidad es $p<1$,
entonces es \emphdef{transitoria}\pause

\begin{thm}[Polya]
    Toda caminata aleatoria simple en $\bZ^d$ es recurrente cuando $d=1,2$ y transitoria
    cuando $d\geq 3$.
\end{thm}
\begin{center}
\begin{tikzpicture}[scale=1.4]
    \draw[gray!50,->,>=latex,line width=1.5pt] (-1.5,0) -- (4,0);
    \draw[gray!50,->,>=latex,line width=1.5pt] (0,-1.5) -- (0,2);
    \filldraw[black] (4,2) circle(0pt)
    node[anchor=north east] {$\bZ^2$};

    \filldraw[black] (0,0) circle(1pt)
    node[anchor=north] {};
    \filldraw[black] (1,0) circle(1pt)
    node[anchor=north] {};
    \filldraw[black] (2,0) circle(1pt)
    node[anchor=north] {};
    \filldraw[black] (3,0) circle(1pt)
    node[anchor=north] {};
    \filldraw[black] (0,1) circle(1pt)
    node[anchor=north] {};
    \filldraw[black] (1,1) circle(1pt)
    node[anchor=north] {};
    \filldraw[black] (2,1) circle(1pt)
    node[anchor=north] {};
    \filldraw[black] (-1,0) circle(1pt)
    node[anchor=north] {};
    \filldraw[black] (-1,1) circle(1pt)
    node[anchor=north] {};
    \filldraw[black] (-1,-1) circle(1pt)
    node[anchor=north] {};
    \filldraw[black] (0,-1) circle(1pt)
    node[anchor=north] {};
    \filldraw[black] (1,-1) circle(1pt)
    node[anchor=north] {};
    \filldraw[black] (3,1) circle(1pt)
    node[anchor=north] {};
    \filldraw[black] (2,-1) circle(1pt)
    node[anchor=north] {};
    \filldraw[black] (3,-1) circle(1pt)
    node[anchor=north] {};

    \draw (0,0) -- (0,1) ;
    \filldraw[black] (0,0.5) circle(0pt)
    node[anchor=east] {\large \mathcolor{blue}{$\frac{1}{4}$}};
    \draw (0,0) -- (1,0) ;
    \filldraw[black] (0.5,0) circle(0pt)
    node[anchor=north] {\large \mathcolor{blue}{$\frac{1}{4}$}};
    \draw[dotted] (1,0) -- (2,0) ;
    \draw[dotted] (2,0) -- (3,0) ;
    \draw (0,1) -- (1,1) ;
    \draw[dotted] (1,1) -- (2,1) ;
    \draw (1,1) -- (1,0) ;
    \draw[dotted] (2,1) -- (2,0) ;
    \draw (-1,1) -- (0,1) ;
    \draw (-1,0) -- (0,0) ;
    \draw (0,0) -- (0,-1) ;
    \draw (0,-1) -- (1,-1) ;
    \draw[dotted] (0,-1) -- (0,-1.25) ;
    \draw (1,-1) -- (1,0) ;
    \draw[dotted] (1,-1) -- (1,-1.25) ;
    \draw (-1,-1) -- (0,-1) ;
    \draw (-1,1) -- (-1,0) ;
    \draw (-1,0) -- (-1,-1) ;
    \draw[dotted] (-1,-1) -- (2,-1);
    \draw[dotted] (-1,-1) -- (-1.25,-1);
    \draw[dotted] (-1,-1) -- (-1,-1.25);
    \draw[dotted] (2,-1) -- (3,-1);
    \draw[dotted] (2,-1) -- (2,-1.25);
    \draw[dotted] (2,-1) -- (2,0);
    \draw[dotted] (3,-1) -- (3,0);
    \draw[dotted] (3,-1) -- (3,-1.25);
    \draw[dotted] (3,-1) -- (3.5,-1);
    \draw[dotted] (3,0) -- (3,1);
    \draw[dotted] (3,1) -- (3,1.5);
    \draw[dotted] (3,0) -- (3.5,0);
    \draw[dotted] (2,1) -- (3,1);
    \draw[dotted] (3,1) -- (3.5,1);
    \draw[dotted] (2,1) -- (2,1.5);
    \draw[dotted] (1,1) -- (1,1.5);
    \draw[dotted] (0,1) -- (0,1.5);
    \draw[dotted] (-1,1) -- (-1,1.5);
    \draw[dotted] (-1,1) -- (-1.25,1);
    \draw[dotted] (-1,0) -- (-1.25,0);


\end{tikzpicture}
\end{center}
\end{frame}

\begin{frame}{Redes eléctricas}
\emphdef{Red eléctrica}: $(\Gamma,\omega)$ gráfica no dirigida con pesos.
\bigskip

\begin{center}
\begin{tabular}{m{6cm} m{6cm}}
\begin{tikzpicture}[baseline={([yshift=-.5ex]current bounding box.center)}]
    \node[my state] (s) at (0,0) {$s$};
    \node[my state] (3) at (2.5,0) {$3$};
    \node[my state] (2) at (1.5,2) {$2$};
    \node[my state] (t) at (4,2) {$t$};

    \path[every node]
    (s) edge[-] node[left] {\mathcolor{blue}{$\omega_{s2}$}} (2);
    \path[every node]
    (s) edge[-] node[below] {\mathcolor{blue}{$\omega_{s3}$}} (3);
    \path[every node]
    (s) edge[-] node[below] {} (t);
    \path[every node]
    (2) edge[-] node[above] {} (t);
    \path[every node]
    (3) edge[-] node[below] {} (t);
\end{tikzpicture}
    &
    $\begin{aligned}[t]
        \phi&\colon V\rightarrow \bR\\
        i&\colon V\times V \rightarrow\bR
    \end{aligned}$
    \\
    \multicolumn{1}{p{5cm}}{\leavevmode\newline\newline $s$: fuente\newline $t$: sumidero}
    &
    \multicolumn{1}{p{5cm}}{\leavevmode\newline\newline $\phi$: voltaje\newline $i$: corriente}
\end{tabular}
\end{center}
\end{frame}

\begin{frame}{Voltajes y corrientes}
\emphdef{Ley de voltajes}: Si $v_1,\dots,v_k,v_{k+1}=v_1\in V$, entonces
\[
    \sum_{j=1}^k \phi(v_j)-\phi(v_{j+1}) = 0
\]
\bigskip

\emphdef{Ley de corrientes}: $\forall u\in V\setminus\{s,t\}$
\[
    \sum_{v\in V}i_{uv}=0
\]

\bigskip
\bigskip
\emphdef{Ley de Ohm}: $\forall u,v\in V$
\[
    i_{uv}=\omega_{uv}(\phi(v)-\phi(u))
\]
\end{frame}

\begin{frame}{Funciones armónicas}
Para una función armónica en un abierto $A\subset\bC$,
\[
    h(a)=\frac{1}{2\pi}\int_0^{2\pi}h(a+re^{i\theta})\,d\theta.
\]
Nos fijamos en la cadena de Markov con matriz de transición
\[
    P_{vu}=\frac{w_{vu}}{W_u},\qquad\text{donde}\qquad
    W_u=\sum_{v\in v}w_{uv}.
\]

\bigskip
\bigskip

Si $U\subset V$, $f\colon V\rightarrow \bR$ es \emphdef{armónica} en $U$ si
\[
    f(u)=\sum_{v\in V}P_{vu}f(v),\quad u\in U
\]


\end{frame}

\begin{frame}{Problema de Dirichlet discreto}
Definir una estructura de red eléctrica es equivalente a
resolver un problema de Dirichlet discreto
cuando la frontera es $\{s,t\}$

\bigskip
\bigskip

\emphdef{Problema de Dirichlet discreto}: Dados $s,t\in V$ y $\alpha,\beta\in\bR$,
entonces existe una única función armónica $\phi\colon V\rightarrow\bR$ 
en $V\setminus\{s,t\}$ tal que $\phi(s)=\alpha$ y $\phi(t)=\beta$.

\bigskip
\bigskip

Obtener una solución problema de Dirichlet discreto da lugar a una estructura de red
eléctrica a través de la ley de Ohm:

\bigskip

\emphdef{Ley de Ohm}: $\forall u,v\in V$
\[
    i_{uv}=\omega_{uv}(\phi(v)-\phi(u))
\]

\end{frame}


\begin{frame}{Ejemplo básico de red eléctrica}
\begin{equation*}
\begin{tikzpicture}[baseline={([yshift=-.5ex]current bounding box.center)}]
    \node[my state] (s) at (0,0) {$s$};
    \node[my state] (1) at (2,0) {$1$};
    \node[my state] (t) at (4,0) {$t$};

    \path[every node]
    (s) edge[-] node[above] {\mathcolor{blue}{$1/2$}} (1);
    \path[every node]
    (1) edge[-] node[above] {\mathcolor{blue}{$1/2$}} (t);
\end{tikzpicture},
\qquad\qquad
P=
\begin{blockarray}{cccc}
& \mathcolor{blue}{s} & \mathcolor{blue}{1} & \mathcolor{blue}{t}\\
\begin{block}{c(ccc)}
\mathcolor{blue}{s} & 0 & 1/2 & 0 \\
\mathcolor{blue}{1} & 1 & 0   & 1 \\
\mathcolor{blue}{t} & 0 & 1/2 & 0 \\
\end{block}
\end{blockarray}
\end{equation*}

%\begin{equation*}
%\begin{tikzpicture}[baseline={([yshift=-.5ex]current bounding box.center)}]
%    \node[my state] (s) at (0,0) {$s$};
%    \node[my state] (1) at (2,0) {$1$};
%    \node[my state] (t) at (4,0) {$t$};
%
%    \path[every node]
%    (1) edge[->, bend right] node[above] {$1/2$} (s);
%    \path[every node]
%    (s) edge[->, bend right] node[below] {$1$} (1);
%    \path[every node]
%    (1) edge[->, bend left] node[above] {$1/2$} (t);
%    \path[every node]
%    (t) edge[->, bend left] node[below] {$1$} (1);
%\end{tikzpicture}
%\end{equation*}


Un voltaje (función armónica) $\phi$ debe satisfacer
\[
    \phi(1) = P_{s1}\phi(s) + P_{t1}\phi(t) = 1/2\cdot\phi(s) + 1/2\cdot\phi(t)
\]
Condiciones de frontera $\phi(s)=0$ y $\phi(t)=1$ implican $\phi(1)=1/2$

\pause

Aplicando la ley de Ohm,
\[
    i = 
    \begin{blockarray}{cccc}
    & \mathcolor{blue}{s} & \mathcolor{blue}{1} & \mathcolor{blue}{t}\\
    \begin{block}{c(ccc)}
    \mathcolor{blue}{s} & 0 & 1/4 & 0 \\
    \mathcolor{blue}{1} & -1/4 & 0 & 1/4 \\
    \mathcolor{blue}{t} & 0 & -1/4 & 0\\
    \end{block}
    \end{blockarray}
\]

\end{frame}

\begin{frame}{Problema de Dirichlet discreto}
Se cumplen las siguientes afirmaciones
\begin{itemize}
    \item
    Cualquier combinación lineal de funciones armónicas también es armónica

    \item
    Si $\psi$ es una función armónica que satisface $\psi(s)=\psi(t)=0$,
    entonces $\psi$ es idénticamente cero.
    
    \item
    Cualquiera dos soluciones al problema de Dirichlet discreto son 
    necesariamente iguales
\end{itemize}

\bigskip\pause

\begin{thm}
    Siempre existe una solución al problema de Dirichlet discreto y ésta es única.
\end{thm}

\end{frame}

\begin{frame}{Resistencia efectiva}
\emphdef{Ley de resistencias en paralelo}: Las redes eléctricas siguientes
tienen la misma resistencia efectiva
\begin{equation*}
    \begin{tikzpicture}[baseline={([yshift=-.5ex]current bounding box.center)}]
        \node[my state] (s) at (0,0) {$s$};
        \node[my state] (1) at (2,0) {$1$};
        \node[my state] (2) at (4,0) {$2$};
        \node[my state] (t) at (6,0) {$t$};

        \path[every node]
        (s) edge[-] node[above] {} (1);
        \path[every node]
        (1) edge[-,bend left] node[above] {\mathcolor{blue}{$r_1$}} (2);
        \path[every node]
        (1) edge[-,bend right] node[below] {\mathcolor{blue}{$r_2$}} (2);
        \path[every node]
        (2) edge[-] node[below] {} (t);
    \end{tikzpicture},
    \qquad 
    \begin{tikzpicture}[baseline={([yshift=0ex]current bounding box.center)}]
        \node[my state] (s) at (0,0) {$s$};
        \node[my state] (1) at (2,0) {$1$};
        \node[my state] (2) at (4,0) {$2$};
        \node[my state] (t) at (6,0) {$t$};

        \path[every node]
        (s) edge[-] node[above] {} (1);
        \path[every node]
        (1) edge[-] node[below] {\mathcolor{blue}{$\frac{r_1r_2}{r_1+r_2}$}} (2);
        \path[every node]
        (2) edge[-] node[below] {} (t);
    \end{tikzpicture}
\end{equation*}

\bigskip
\bigskip\pause

\emphdef{Ley de resistencias en serie}: Las redes eléctricas siguientes
tienen la misma resistencia efectiva
\begin{equation*}
    \begin{tikzpicture}[baseline={([yshift=-.5ex]current bounding box.center)}]
        \node[my state] (s) at (0,0) {$s$};
        \node[my state] (1) at (2,0) {$1$};
        \node[my state] (2) at (4,0) {$2$};
        \node[my state] (3) at (6,0) {$3$};
        \node[my state] (t) at (8,0) {$t$};

        \path[every node]
        (s) edge[-] node[above] {} (1);
        \path[every node]
        (1) edge[-] node[above] {\mathcolor{blue}{$r_1$}} (2);
        \path[every node]
        (2) edge[-] node[above] {\mathcolor{blue}{$r_2$}} (3);
        \path[every node]
        (3) edge[-] node[below] {} (t);
    \end{tikzpicture},
    \qquad 
    \begin{tikzpicture}[baseline={([yshift=-0.5ex]current bounding box.center)}]
        \node[my state] (s) at (0,0) {$s$};
        \node[my state] (1) at (2,0) {$1$};
        \node[my state] (3) at (4,0) {$3$};
        \node[my state] (t) at (6,0) {$t$};

        \path[every node]
        (s) edge[-] node[above] {} (1);
        \path[every node]
        (1) edge[-] node[above] {\mathcolor{blue}{$r_1+r_2$}} (2);
        \path[every node]
        (2) edge[-] node[below] {} (t);
    \end{tikzpicture}
\end{equation*}

\end{frame}

\begin{frame}{Flujos y energía disipada}
\emphdef{$\bm{(s,t)}$-flujo}: es una función $j\colon V\times V\rightarrow\bR$ tal
que
\begin{itemize}
    \item
    $j_{uv}=-j_{vu}$
    \item
    $j_{uv}=0$ si $\{u,v\}\not\in V$
    \item
    si $u\neq s,t$, entonces $J_u=\sum_{v\in V}j_{uv}=0$
\end{itemize}

La corriente eléctrica $i$ es un ejemplo de $(s,t)$-flujo. Podemos tener
varios $(s,t)$-flujos pero sólo una corriente eléctrica.

\bigskip

\emphdef{Energía disipada} de un $(s,t)$-flujo $j$:
\[
    E(j) = \sum_{e_\in E}j_e^2 r_e = \frac{1}{2}\sum_{u,v\in V}j_{uv}^2 r_{uv}
\]

\end{frame}


\begin{frame}{Resistencia efectiva}
En una red eléctrica con voltaje $\phi$ y corriente $i$ se puede demostrar que
\[
    (\phi(t)-\phi(s))\cI_s = E(i)
\]

\emphdef{Resistencia efectiva}: 
\[
    \Reff =\frac{\phi(t)-\phi(s)}{\cI_s} =\cdots = \frac{E(i)}{\cI_s^2} 
\]
Está alineada a la ley de Ohm. ($V=RI$.)
\end{frame}

\begin{frame}{Principio de Rayleigh}
%\begin{thm}[Principio de Thomson]
%    El $(s,t)$-flujo que satisface las leyes de Kirchhoff es el único flujo
%    $i$ que minimiza a la energía disipada.
%\end{thm}
%
%Podemos tener varios $(s,t)$-flujos pero sólo una corriente en la red eléctrica.
%Fijada a través del voltaje por ser solución al problema de Dirichlet discreto.


\begin{thm}[Principio de Rayleigh] 
    La resistencia efectiva de una
    red eléctrica es creciente como función de las resistencias $r_e$ de las
    aristas ($e\in E$).
\end{thm}
\bigskip
\bigskip

\begin{tabular}{c c c}
    \begin{tikzpicture}[baseline={([yshift=-.5ex]current bounding box.center)}]
        \node[my state] (s) at (0,0) {$s$};
        \node[my state] (1) at (2,0) {$1$};
        \node[my state] (2) at (4,0) {$2$};
        \node[my state] (3) at (6,0) {$3$};
        \node[my state] (t) at (8,0) {$t$};
        \path[every node]
        (s) edge[-] node[above] {} (1);
        \path[every node]
        (1) edge[-] node[above] {\mathcolor{blue}{$r_1$}} (2);
        \path[every node]
        (2) edge[-] node[above] {\mathcolor{blue}{$r_2$}} (3);
        \path[every node]
        (3) edge[-] node[below] {} (t);
    \end{tikzpicture}
    &$\qquad$
    &\begin{tikzpicture}[baseline={([yshift=-0.5ex]current bounding box.center)}]
        \node[my state] (s) at (0,0) {$s$};
        \node[my state] (1) at (2,0) {$1$};
        \node[my state] (2) at (4,0) {$2$};
        \node[my state] (t) at (6,0) {$t$};
        \path[every node]
        (s) edge[-] node[above] {} (1);
        \path[every node]
        (1) edge[-] node[above] {\mathcolor{blue}{$r_1$}} (2);
        \path[every node]
        (2) edge[-] node[below] {} (t);
    \end{tikzpicture}\\
    &&\\
    $\Reff$ &$\geq$ &$\Reff'$
\end{tabular}
\end{frame}

\begin{frame}{Resistencia efectiva y recurrencia}
Nos concentramos en la malla $\bZ^d$. Comenzamos por definir resistencia efectiva
partiendo de subgráficas finitas.

Nos fijamos en $\Gamma_n = (V_n, E_n)$. 
\begin{align*}
    \Delta_n &= \{v\in V\colon\ d(0,v)\leq n\}\\
    V_{n} &= \Delta_{n-1}\ \cup\ I_n,
    \quad I_n=\bar{V\setminus\Delta_{n-1}}\\
\end{align*}

\begin{center}
\begin{tabular}{>{\centering\arraybackslash}m{3cm} >{\centering\arraybackslash}m{1cm} >{\centering\arraybackslash}m{3cm} >{\centering\arraybackslash}m{1cm} >{\centering\arraybackslash}m{3cm} >{\centering\arraybackslash}m{3cm}}
%\begin{tabular}{c c c c >{\centering\arraybackslash}m{1cm} }
$\Gamma_1$ & & $\Gamma_2$ & & $\Gamma_3$ & \\
\begin{tikzpicture}
    \draw[gray!50,->,>=latex,line width=1.5pt] (-1.5,0) -- (1.5,0);
    \draw[gray!50,->,>=latex,line width=1.5pt] (0,-1.5) -- (0,1.5);

    \filldraw[black] (0,0) circle(1.5pt)
    node[anchor=north] {};

    \node[anchor=west] at (1,1) (In) {\small $I_1$};
    \filldraw[black] (1,1) circle(1.5pt)
    node[anchor=west] {};
    \draw (0,0) edge[dashed,-] (1,1);
\end{tikzpicture}
& &
\begin{tikzpicture}
    \draw[gray!50,->,>=latex,line width=1.5pt] (-1.5,0) -- (1.5,0);
    \draw[gray!50,->,>=latex,line width=1.5pt] (0,-1.5) -- (0,1.5);

    \draw[black,line width=0.5pt] (-1,0) -- (1,0);
    \draw[black,line width=0.5pt] (0,-1) -- (0,1);

    \filldraw[black] (0,0) circle(1.5pt)
    node[anchor=north] {};

    \filldraw[black] (1,0) circle(1.5pt)
    node[anchor=north] {};
    \filldraw[black] (-1,0) circle(1.5pt)
    node[anchor=north] {};
    \filldraw[black] (0,1) circle(1.5pt)
    node[anchor=north] {};
    \filldraw[black] (0,-1) circle(1.5pt)
    node[anchor=north] {};

    \node[anchor=west] at (1,1) (In) {\small $I_2$};
    \filldraw[black] (1,1) circle(1.5pt)
    node[anchor=west] {};
    \draw (0,1) edge[out=20,in=160,dashed,-] (1,1);
    \draw (-1,0) edge[out=40,in=190,dashed,-] (1,1);
    \draw (0,-1) edge[out=45,in=260,dashed,-] (1,1);
    \draw (1,0) edge[out=60,in=290,dashed,-] (1,1);
\end{tikzpicture}
& &
\begin{tikzpicture}
    \draw[gray!50,->,>=latex,line width=1.5pt] (-1.5,0) -- (1.5,0);
    \draw[gray!50,->,>=latex,line width=1.5pt] (0,-1.5) -- (0,1.5);

    \filldraw[black] (0,0) circle(1.5pt)
    node[anchor=north] {};
    \filldraw[black] (0.5,0) circle(1.5pt)
    node[anchor=north] {};
    \filldraw[black] (-0.5,0) circle(1.5pt)
    node[anchor=north] {};
    \filldraw[black] (0,0.5) circle(1.5pt)
    node[anchor=north] {};
    \filldraw[black] (0,-0.5) circle(1.5pt)
    node[anchor=north] {};
    
    \filldraw[black] (1,0) circle(1.5pt)
    node[anchor=north] {};
    \filldraw[black] (-1,0) circle(1.5pt)
    node[anchor=north] {};
    \filldraw[black] (0,1) circle(1.5pt)
    node[anchor=north] {};
    \filldraw[black] (0,-1) circle(1.5pt)
    node[anchor=north] {};
    \filldraw[black] (0.5,0.5) circle(1.5pt)
    node[anchor=north] {};
    \filldraw[black] (-0.5,0.5) circle(1.5pt)
    node[anchor=north] {};
    \filldraw[black] (0.5,-0.5) circle(1.5pt)
    node[anchor=north] {};
    \filldraw[black] (-0.5,-0.5) circle(1.5pt)
    node[anchor=north] {};

    \draw[black,line width=0.5pt] (0,-1) -- (0,-0.5);
    \draw[black,line width=0.5pt] (-1,0) -- (-0.5,0);
    \draw[black,line width=0.5pt] (0.5,0) -- (1,0);
    \draw[black,line width=0.5pt] (0,0.5) -- (0,1);
    \draw[black,line width=0.5pt] (-0.5,0.5) -- (-0.5,-0.5);
    \draw[black,line width=0.5pt] (-0.5,-0.5) -- (0.5,-0.5);
    \draw[black,line width=0.5pt] (0.5,-0.5) -- (0.5,0.5);
    \draw[black,line width=0.5pt] (0.5,0.5) -- (-0.5,0.5);
    
    \draw[black,line width=0.5pt] (0,0) -- (0,-0.5);
    \draw[black,line width=0.5pt] (0,0) -- (0,0.5);
    \draw[black,line width=0.5pt] (0,0) -- (0.5,0);
    \draw[black,line width=0.5pt] (0,0) -- (-0.5,0);

    \node[anchor=west] at (1,1) (In) {\small $I_3$};
    \filldraw[black] (1,1) circle(1.5pt)
    node[anchor=west] {};
    \draw (0,1) edge[out=20,in=160,dashed,-] (1,1);
    \draw (1,0) edge[out=60,in=290,dashed,-] (1,1);
    \draw (0.5,0.5) edge[out=30,in=210,dashed,-] (1,1);
\end{tikzpicture}
& $\cdots$ \\
$\Reff(1)$ & $\leq$ & $\Reff(2)$ & $\leq$ & $\Reff(3)$ &
\end{tabular}
\end{center}
\end{frame}

\begin{frame}{Resistencia efectiva y recurrencia}
Cada $\Gamma_n$ con estructura de red electrica dada por la funcion armónica
$\phi_n$ solución al problema de Dirichlet
\[
    \phi_n(0)=0,\quad \phi_n(I_n)=1
\]

\bigskip

Entonces, en la malla $\bZ^d$ definimos
\[
    \Reff = \lim_{n\to\infty}\Reff(n)
\]

\end{frame}

\begin{frame}{Resistencia efectiva y recurrencia}
\begin{thm}
    La probabilidad de retorno de la caminata $Z$ a $0$ está dada por
    \[
        \bP(Z_n=0\text{ para algún }n\geq 1| Z_0 =0)=1-\frac{1}{W_0\Reff},
    \]
    donde 
    $
        W_0 = \sum_{v\in V}\omega_{0v}
    $
\end{thm}
\bigskip
Para $d=1,2$ probaremos que $\Reff=\infty$
%Entonces $Z$ es recurrente si y sólo si $\Reff=\infty$ ($d=1,2$).

\bigskip
Para el caso $d\geq 3$ introducimos el concepto de $(0,\infty)$-flujo
\end{frame}


\begin{frame}{$(0,\infty)$-flujos y transitoriedad}
\emphdef{$\bm{(0,\infty)}$-flujo}: Un flujo con fuente pero sin sumidero.
\begin{itemize}
    \item
    Ley de corrientes de Kirchhoff
    \[
        \sum_{v\in V}j_{uv}=0\text{ si }u\neq 0
    \]
    
    \item
    $j_{uv}=-j_{vu}$

    \item
    $j_{uv}=0$ si $\{u,v\}$ no es arista de $\Gamma$
\end{itemize}

\bigskip
\bigskip

\begin{thm}
    La caminata $Z$ es transitoria si y sólo si existe un $(0,\infty)$-flujo
    $j$ en $\Gamma$ con energía disipada finita.
\end{thm}
\end{frame}

\begin{frame}{Demostración del Teorema de Polya}
{\large Caso $d=1$}
Para $d=1$, tenemos

\bigskip
\begin{center}
    \begin{tikzpicture}[baseline={([yshift=-.5ex]current bounding box.center)}]
        \node (d) at (-2,0) {$\cdots$};
        \node[my state] (1) at (0,0) {$-2$};
        \node[my state] (2) at (2,0) {$-1$};
        \node[my state] (3) at (4,0) {$0$};
        \node[my state] (4) at (6,0) {$1$};
        \node[my state] (5) at (8,0) {$2$};
        \node (e) at (10,0) {$\cdots$};

        %\filldraw[black] (-2,0) circle(0pt)
        %node[anchor=west] {$\cdots$};
        %\draw[-] (-1.5,0) to (0,0);
        \path[every node]
        (d) edge[-] node[above] {} (1);
        \path[every node]
        (1) edge[-] node[above] {} (2);
        \path[every node]
        (2) edge[-] node[above] {\mathcolor{blue}{$\frac{1}{2}$}} (3);
        \path[every node]
        (3) edge[-] node[above] {\mathcolor{blue}{$\frac{1}{2}$}} (4);
        \path[every node]
        (4) edge[-] node[above] {} (5);
        \path[every node]
        (5) edge[-] node[above] {} (e);
    \end{tikzpicture}
\end{center}

\bigskip
\bigskip

De la ley de resistencias en serie obtenemos que
\[
    \Reff = \sum_{k=1}^\infty \frac{1}{2} = \infty
\]

\bigskip

Por lo que el teorema se cumple en $\bZ^1$

\end{frame}

\begin{frame}{Demostración del Teorema de Polya...}{\large Caso $d=2$}
En cada $\Gamma_n$ acotamos su $\Reff(n)$ por abajo. En $\Gamma_n$ identificamos
los conjuntos $\partial\Delta_1,\partial\Delta_2,\dots,\partial\Delta_{n-1}$.

\begin{center}
\begin{tabular}{>{\centering\arraybackslash}m{3cm} >{\centering\arraybackslash}m{1cm} >{\centering\arraybackslash}m{3cm} >{\centering\arraybackslash}m{1cm} >{\centering\arraybackslash}m{3cm}}
%\begin{tabular}{c c c c >{\centering\arraybackslash}m{1cm} }
$\Gamma_2$ & & $\Gamma_3$ & & \\
\begin{tikzpicture}
    \draw[gray!50,->,>=latex,line width=1.5pt] (-1.5,0) -- (1.5,0);
    \draw[gray!50,->,>=latex,line width=1.5pt] (0,-1.5) -- (0,1.5);

    \draw[blue,line width=0.5pt] (-1,0) -- (1,0);
    \draw[blue,line width=0.5pt] (0,-1) -- (0,1);

    \filldraw[black] (0,0) circle(1.5pt)
    node[anchor=north] {};

    \filldraw[blue] (1,0) circle(1.5pt)
    node[anchor=north] {};
    \filldraw[blue] (-1,0) circle(1.5pt)
    node[anchor=north] {};
    \filldraw[blue] (0,1) circle(1.5pt)
    node[anchor=north] {};
    \filldraw[blue] (0,-1) circle(1.5pt)
    node[anchor=north] {};

    \node[anchor=west] at (1,1) (In) {\small $I_2$};
    \filldraw[black] (1,1) circle(1.5pt)
    node[anchor=west] {};
    \draw (0,1) edge[out=20,in=160,dashed,-] (1,1);
    \draw (1,0) edge[out=60,in=290,dashed,-] (1,1);

\end{tikzpicture}
& &
\begin{tikzpicture}
    \draw[gray!50,->,>=latex,line width=1.5pt] (-1.5,0) -- (1.5,0);
    \draw[gray!50,->,>=latex,line width=1.5pt] (0,-1.5) -- (0,1.5);
    
    \filldraw[blue] (1,0) circle(1.5pt)
    node[anchor=north] {};
    \filldraw[blue] (-1,0) circle(1.5pt)
    node[anchor=north] {};
    \filldraw[blue] (0,1) circle(1.5pt)
    node[anchor=north] {};
    \filldraw[blue] (0,-1) circle(1.5pt)
    node[anchor=north] {};
    \filldraw[blue] (0.5,0.5) circle(1.5pt)
    node[anchor=north] {};
    \filldraw[blue] (-0.5,0.5) circle(1.5pt)
    node[anchor=north] {};
    \filldraw[blue] (0.5,-0.5) circle(1.5pt)
    node[anchor=north] {};
    \filldraw[blue] (-0.5,-0.5) circle(1.5pt)
    node[anchor=north] {};

    \draw[red,line width=0.5pt] (0,0) -- (0,-0.5);
    \draw[red,line width=0.5pt] (0,0) -- (0,0.5);
    \draw[red,line width=0.5pt] (0,0) -- (0.5,0);
    \draw[red,line width=0.5pt] (0,0) -- (-0.5,0);

    \draw[blue,line width=0.5pt] (0,-1) -- (0,-0.5);
    \draw[blue,line width=0.5pt] (-1,0) -- (-0.5,0);
    \draw[blue,line width=0.5pt] (0.5,0) -- (1,0);
    \draw[blue,line width=0.5pt] (0,0.5) -- (0,1);
    \draw[blue,line width=0.5pt] (-0.5,0.5) -- (-0.5,-0.5);
    \draw[blue,line width=0.5pt] (-0.5,-0.5) -- (0.5,-0.5);
    \draw[blue,line width=0.5pt] (0.5,-0.5) -- (0.5,0.5);
    \draw[blue,line width=0.5pt] (0.5,0.5) -- (-0.5,0.5);
    
    \node[anchor=west] at (1,1) (In) {\small $I_3$};
    \filldraw[black] (1,1) circle(1.5pt)
    node[anchor=west] {};
    \draw (0,1) edge[out=20,in=160,dashed,-] (1,1);
    \draw (1,0) edge[out=60,in=290,dashed,-] (1,1);
    \draw (0.5,0.5) edge[out=30,in=210,dashed,-] (1,1);

    \filldraw[black] (0,0) circle(1.5pt)
    node[anchor=north] {};
    
    \filldraw[red] (0.5,0) circle(1.5pt)
    node[anchor=north] {};
    \filldraw[red] (-0.5,0) circle(1.5pt)
    node[anchor=north] {};
    \filldraw[red] (0,0.5) circle(1.5pt)
    node[anchor=north] {};
    \filldraw[red] (0,-0.5) circle(1.5pt)
    node[anchor=north] {};

\end{tikzpicture}
& & $\cdots$ \\
%$\Reff(1)$ & $\leq$ & $\Reff(2)$ & $\leq$ & 
\end{tabular}
\end{center}

De manera que en $\Gamma_n$ obtenemos la red
\begin{center}
    \begin{tikzpicture}
        \tikzstyle{dots} = []
        \node[my state] (0) at (0,0) {$0$};
        \node[my state] (1) at (2,0) {$1$};
        \node[my state] (2) at (4,0) {$2$};
        \node[dots] (3) at (6,0) {$\cdots$};

        \path[every node]
        (0) edge[-,bend left=20] node[above] {} (1);
        \path[every node]
        (0) edge[-,bend left=40] node[above] {\small \mathcolor{blue}{$4$}} (1);
        \path[every node]
        (0) edge[-,bend right=20] node[below] {} (1);
        \path[every node]
        (0) edge[-,bend right=40] node[below] {$4$} (1);
        \path[every node]
        (1) edge[-,bend left=40] node[below] {$\vdots$} (2);
        \path[every node]
        (1) edge[-,bend left=40,line width=0pt] node[above] 
        {\small \mathcolor{blue}{$4$}} (2);
        \path[every node]
        (1) edge[-,bend right=40] node[below] {$12$} (2);
        \path[every node]
        (2) edge[-,bend left=40] node[below] {$\vdots$} (3);
        \path[every node]
        (2) edge[-,bend left=40,line width=0pt] node[above] 
        {\small \mathcolor{blue}{$4$}} (3);
        \path[every node]
        (2) edge[-,bend right=40] node[below] {$8k-4$} (3);
    \end{tikzpicture}
\end{center}
\end{frame}

\begin{frame}{Demostración del Teorema de Polya...}{\large Caso $d=2$}
Para cada $n$, tenemos que
\[
    \frac{4}{4}+\frac{4}{12}+\cdots=\sum_{k=1}^n \frac{4}{8k-4} \leq \Reff(n)
\]
\bigskip
Por lo que
\[
    \Reff = \infty
\]
y el teorema se cumple en $\bZ^2$
\end{frame}

\begin{frame}{Demostración del Teorema de Polya...}{\large Caso $d=3$}
Queremos construir un $(0,\infty)$-flujo con energía finita.

\begin{center}
\begin{tikzpicture}
    \draw (0,0) circle(1cm);
    \draw[solid,black!30] (-1,0) arc [
        start angle=180,
        end angle=360,
        x radius=1cm,
        y radius=0.25cm
    ];
    \draw[gray!40,->,>=latex,line width=1.5pt] (0,0) -- (8,0);
    \draw[gray!40,->,>=latex,line width=1.5pt] (0,0) -- (0,2);
    \draw[gray!40,->,>=latex,line width=1.5pt] (0,0) -- (-1.5,-1.5);

    \filldraw[black] (6,0) circle(1pt)
    node[anchor=north] {$u$};
    \filldraw[black] (5,0) circle(1pt)
    node[anchor=north] {$v$};
    \draw (7.25,1.25) -- (5.25,1.25);
    \draw (6.75,0.75) -- (4.75,0.75);
    \draw (6.75,0.75) -- (7.25,1.25);
    \draw (5.25,1.25) -- (4.75,0.75);
    \draw (6.75,0.75) --(6.75,-1.25); 
    \draw (6.75,-1.25) --(4.75,-1.25); 
    \draw (4.75,-1.25) -- (4.75,0.75);
    \draw (6.75,-1.25) -- (7.25,-0.75);
    \draw (7.25,-0.75) -- (7.25,1.25);
    \draw[dotted,line width=1pt] (4.75,-1.25) -- (5.25,-0.75);
    \draw[dotted,line width=1pt] (5.25,-0.75) -- (5.25,1.25);
    \draw[dotted,line width=1pt] (5.25,-0.75) -- (7.25,-0.75);

    \draw[dashed,line width=1pt,blue!40] (0,0) -- (4.75,0.75);
    \draw[dashed,line width=1pt,blue!40] (0,0) -- (4.75,-1.25);
    \draw[dashed,line width=1pt,blue!40] (0,0) -- (5.25,1.25);
    \draw[dashed,line width=1pt,blue!40] (0,0) -- (5.25,-0.75);

\end{tikzpicture}
\end{center}
\bigskip
Con el signo de 
\[
    \langle u+v,u-v\rangle
\]

\end{frame}

%\begin{frame}{Demostración del Teorema de Polya...}{\large Caso $d=3$}
%$j$ es un $(0,\infty)$-flujo no trivial ($J_0\neq 0$)
%\begin{itemize}
%    \item
%    Ley de corrientes de Kirchhoff
%    \[
%        \sum_{v\in V}j_{uv}=0\text{ si }u\neq 0
%    \]
%    
%    \item
%    $j_{uv}=-j_{vu}$
%
%    \item
%    $j_{uv}=0$ si $\{u,v\}$ no es arista de $\Gamma$
%\end{itemize}
%\end{frame}

\begin{frame}{Demostración del Teorema de Polya...}{Caso $d=3$}
Sólo falta ver que tiene energía finita.

\bigskip

%Cuando $\|u\|$ es grande, $\|(u_1-1,u_2,u_3-1)\|\approx\|u\|$, de manera que
%\[
%    \left\|\frac{(u_1-1,u_2,u_3-1)}{\|(u_1-1,u_2,u_3-1)\|}
%    -\frac{(u_1+1,u_2,u_3-1)}{\|(u_1+1,u_2,u_3-1)\|}\right\|\approx
%    \frac{\|(2,0,0)\|}{\|u\|}=\frac{2}{\|u\|}
%\]
%
%\bigskip
%
%y luego

Podemos acotar a $|j_{uv}|$ por arriba:
\[
    |j_{uv}|=\left|\int_{\Pi_v}\,dS\right|\leq
    \frac{A'}{\|u\|^2} \leq \frac{A}{d(0,u)^2}
\]

Ahora la cantidad de vértices con $d(0,u)=n$. (En el primer octante, es lo mismo
que contar particiones de $n$.) Entonces
\[
    |\{u\colon\ d(0,u)=n\}| \leq Bn^2
\]

\end{frame}

\begin{frame}{Demostración del Teorema de Polya...}{\large Caso $d=3$}

Acotemos la energía por arriba.
\begin{align*}
    E(j)=\sum_{u\neq 0}\sum_{v\sim u}j_{uv}^2r_{uv}
    &=\sum_{u\neq 0}\sum_{v\sim u}6j_{uv}^2\\
    &\leq \sum_{d(0,u)\geq 1}\sum_{v\sim u}6\Big(\frac{A}{d(0,u)^2}\Big)^2\\
    &=\sum_{n=1}^\infty \sum_{v\sim u}6\Big(\frac{A}{n^2}\Big)^2
    \leq\sum_{n=1}^\infty 6Bn^2\Big(\frac{A}{n^2}\Big)^2 <\infty\\
\end{align*}

\end{frame}

\begin{frame}{Resumen de la segunda parte}
\begin{itemize}
    \item
    Una estructura de red eléctrica es equivalente a resolver un problema de 
    Dirichlet discreto.

    \bigskip

    \item
    Si las resistencias bajan, también baja la resistencia efectiva.

    \bigskip

    \item
    Para dimensiones $d=1,2$ nos basamos en lo siguiente
    \[
        \bP(Z\text{ regresa a }0) = 1-\frac{1}{\text{cte}\cdot\Reff}
    \]

    \bigskip
    
    \item
    Para dimensiones $d\geq 3$ construimos un $(0,\infty)$-flujo no trivial
    con energía finita.
\end{itemize}
\end{frame}

\begin{frame}{Teorema de Polya vía funciones especiales}
Comenzamos por contar bucles y bucles indescomponibles. \pause
Introducimos las probabilidades $p_n$ y $q_n$ de manera combinatoria.
$q_n$ es la probabilidad de regresar al origen en $n$ pasos.
$p_n$ es la probabilidad de regresar al origen por primera vez en $n$ pasos. \pause
\[
    P(z)=\sum_{n=0}^\infty p_nz^n,\ 
    Q(z)=\sum_{n=0}^\infty q_nz^n
    \quad \text{satisfacen} \quad
    PQ = Q-1
    \quad \text{en}\quad
    \bR[[z]]
\]

Interpretamos esta identidad como funciones analíticas para pasar de argumentos
combinatorios al análisis de cierto límite. \pause
Al final, el teorema es equivalente a estudiar la convergencia 
o divergencia de la integral
\[
    \int_1^\infty t^{-d/2}\,dt
\]
\end{frame}

%\begin{frame}{Vía funciones especiales}
%\begin{itemize}
%    \item[$\rightarrow$]
%    Funciones generadoras 
%
%    \bigskip
%
%    \item[$\rightarrow$]
%    Transformada de Borel 
%
%    \bigskip
%
%    \item[$\rightarrow$]
%    Expresión integral de la función de Bessel modificada del primer tipo
%    de grado cero
%
%    \bigskip
%    
%    \item[$\rightarrow$]
%    Principio de Laplace
%\end{itemize}
%
%\bigskip
%\bigskip
%
%Si $d$ es tal que la
%integral \[ \int_1^\infty t^{-d/2}\,dt \] diverge, entonces una caminata aleatoria
%simple en $\bZ^d$ será recurrente, y será transitoria en otro caso.
%\end{frame}
%
%
%\begin{frame}{Funciones generadoras}{Ordinarias}
%
%\[
%    a_0,a_1,a_2,\dots \quad\longmapsto\quad\sum_{n=0}^\infty a_nz^n
%\]
%
%
%\begin{thm}\label{thm:producto_de_funciones_generadoras_ordinarias}
%    Dada dos estructuras combinatorias $A$ y $B$, sean $a_n$ y $b_n$ el número total de
%    formas posibles de dotar a un conjunto de $n$ elementos con las estructuras
%    $A$ y $B$, respectivamente. Por otro
%    lado, sea $c_n$ el número de formas de partir al conjunto
%    $\{1,\dots,n\}$ en dos intervalos $S=\{1,\dots,i\}$, $T=\{i+1,\dots,n\}$,
%    posiblemente vacíos, y dotarlos de las estructuras $A$ a $S$ y $B$ a $T$.
%    Si $A(x)$, $B(x)$ y $C(x)$ denotan las funciones generadoras ordinarias de las
%    sucesiones $(a_n)$, $(b_n)$ y $(c_n)$, respectivamente, entonces 
%    \[
%        A(x)B(x)=C(x).
%    \]
%\end{thm}
%
%\end{frame}
%
%\begin{frame}{Funciones generadoras}{Exponenciales}
%\[
%    a_0,a_1,a_2,\dots \quad\longmapsto\quad\sum_{n=0}^\infty \frac{a_n}{n!}z^n
%\]
%
%\begin{thm}\label{thm:producto_de_funciones_generadoras_exponenciales}
%    Sean $A$ y $B$ dos estructuras combinatorias. Sean $a_n$ y $b_n$ el número
%    total de formas posibles de dotar a un conjunto de $n$ elementos con las
%    estructuras $A$ y $B$, respectivamente. Por otro lado, sea $c_n$ el número
%    de formas de expresar al conjunto $\{1,\dots,n\}$ como la
%    unión de dos conjuntos $S$, $T$ disjuntos y dotarlos de las estructuras $A$
%    a $S$ y $B$ a $T$.  Si $A(x)$, $B(x)$ y $C(x)$ denotan las funciones
%    generadoras exponenciales de las sucesiones $(a_n)$, $(b_n)$ y $(c_n)$,
%    respectivamente, entonces
%    \[
%        A(x)B(x)=C(x).
%    \]
%\end{thm}
%\end{frame}
%
%\begin{frame}{Transformada de Borel}
%
%La transformada integral
%\[
%    (\cB f)(z) = \int_0^\infty f(tz)e^{-t}\,dt
%    ,
%\]
%es llamada la \emphdef{transformada de Borel} $f$.  
%
%\bigskip
%\bigskip
%\bigskip
%
%\begin{thm}
%    Si $f$ es la función generadora exponencial de $(a_n)$, entonces
%    $\cB f$ es su función generadora ordinaria.
%\end{thm}
%
%\end{frame}
%
%\begin{frame}{Funciones de Bessel modificadas}
%\emphdef{Ecuación de Bessel} de orden $\alpha$
%\[
%    \frac{d^2y}{dx^2}+\frac{1}{x}\frac{dy}{dx}+\Big(1-\frac{\alpha^2}{x^2}\Big)y=0
%\]
%Cuando $\Re\alpha>-1/2$, tenemos soluciones $J_\alpha$. Se puede probar
%que
%\begin{align*}
%    I_\alpha\colon\bR&\longrightarrow\bR\\
%    x&\longmapsto e^{-\alpha\pi i/2}J_\alpha (xe^{\pi i/2})
%\end{align*}
%Llamada la \emphdef{función de Bessel modificada} del primer tipo, de orden $\alpha$.
%
%\end{frame}
%
%
%
%\begin{frame}
%
%Tenemos estas representaciones
%\alt<2>{
%\begin{itemize}
%    \item en serie de potencias
%    \[
%        I_{0}(x)=
%        \sum_{k=0}^\infty \frac{(x/2)^{2k}}{k!k!}.
%    \]
%
%    \item
%    integral
%    \[
%        I_{0} (x)
%        =\frac{1}
%        {\pi}
%        \int_0^\pi e^{x\cos\theta}\,d\theta
%    \]
%\end{itemize}
%}{
%\begin{itemize}
%    \item en serie de potencias
%    \[
%        I_{\mathcolor{red}{\alpha}}(x)=
%        \sum_{k=0}^\infty \frac{(x/2)^{2k+
%        \mathcolor{red}{\alpha}}}{k!\Gamma(k+
%        \mathcolor{red}{\alpha}+1)}.
%    \]
%
%    \item
%    integral
%    \[
%        I_{\mathcolor{red}{\alpha}} (x)
%        =\frac{(x/2)^{\mathcolor{red}{\alpha}}}
%        {\sqrt\pi\Gamma(\mathcolor{red}{\alpha}+1/2)}
%        \int_0^\pi e^{x\cos\theta}(\sin\theta)^{2\mathcolor{red}{\alpha}}\,d\theta
%    \]
%\end{itemize}
%}
%
%\end{frame}
%
%\begin{frame}{Principio de Laplace}
%
%\[
%    I(t)=\int_a^b e^{-t\phi(x)}f(x)\,dx,\quad t\in(0,\infty),
%\]
%
%\bigskip
%\bigskip
%
%\begin{thm}[Princpio de Laplace]\label{thm:principio_de_laplace}
%    Supongamos que $a<b$ y $\phi,f\colon[a,b]\rightarrow\bR$ son suaves en $[a,b]$.
%    Si además $\phi$ posee un mínimo global en $x_0\in(a,b)$, entonces
%    la función $I$ definida arriba satisface
%    \[
%        I(t)\sim \mathrm{cte}\cdot t^{-1/2}e^{-t\phi(x_0)}
%    \]
%    cuando $t\to\infty$. 
%\end{thm}
%
%\end{frame}
%
%\begin{frame}{Demostración del T. de Polya}
%
%${\bm p}$: la probabilidad de una caminata en $\bZ^d$ que comienza en $0$
%regrese al origen
%
%\bigskip
%\begin{block}{Objetivo}
%Mostrar que $p=1$ cuando $d=1,2$ y que $p<1$ cuando
%$d\geq 3$
%\end{block}
%
%\bigskip
%
%${\bm p_n}$: la probabilidad de que una caminata regrese al origen
%por primera vez después de $n$ pasos
%
%\[
%    p = \sum_{n=0}^\infty p_n = \sum_{n=1}^\infty p_n, \qquad p_0=0
%\]
%
%
%${\bm q_n}$: la probabilidad de que una caminata regrese al origen
%después de $n$ pasos
%\onslide<2>
%\mathcolor{red}{
%\begin{align*}
%    Q(z) = \sum_{n=0}^\infty q_nz^n = 1 + \sum_{n=1}^\infty q_nz^n,\quad q_0=1,
%    \qquad
%    P(z)=\sum_{n=0}^\infty p_nz^n
%\end{align*}
%}
%
%
%\end{frame}
%
%\begin{frame}{Relación entre $P$ y $Q$}
%\emphdef{Bucle}: caminata que comiena y termina en el origen,
%\emphdef{no se descompone} si no se puede expresar como concatenación
%de bucles no triviales
%
%
%${\bm \ell_n}$: cantidad de bucles de longitud $n$
%
%
%${\bm r_n}$: cantidad de bucles que no se descomponen de longitud $n$
%
%\pause
%
%Todo bucle se expresa como concatenación de un bucle indescomponible seguido
%de otro bucle posiblemente trivial
%
%\[
%    \ell_n = \sum_{k=0}^nr_k\ell_{n-k}, \qquad n\geq 1
%\]
%
%\bigskip
%
%Al dividir entre $(2d)^n$, el número de caminatas de longitud $n$ en $\bZ^d$
%
%\[
%    q_n = \sum_{k=0}^np_kq_{n-k}, \qquad n\geq 1 
%    \quad\qquad\Longrightarrow\quad\qquad
%    P(z)Q(z) = Q(z)-1\ \text{en }\bR[[z]]
%\]
%
%\end{frame}
%
%\begin{frame}{$P$ y $Q$ como funciones analíticas en $\bD\subset\bC$}
%Tenemos $0\leq p_n\leq q_n\leq 1$ y entonces
%\[
%    \text{radio de convergencia} = \frac{1}{\limsup p_n} \geq 1
%\]
%
%\bigskip
%
%\[
%    P,Q\colon\bD\rightarrow\bC
%\]
%
%\bigskip
%
%Cuando $z\in[0,1)$, $Q(z)\geq 1$. De manera que si $0\leq z< 1$
%\[
%    P(z) = 1-\frac{1}{Q(z)}
%\]
%
%\bigskip
%
%Del teorema de series de potencias de Abel
%\[
%    p = \lim_{\substack{z\to 1\\z\in[0,1)}}P(z)
%    =1-\frac{1}{\lim_{\substack{z\to 1\\z\in[0,1)}}Q(z)}
%\]
%
%\end{frame}
%
%\begin{frame}{Contando bucles en $\bZ^d$}
%Estamos analizando $Q$
%\[
%    L(z) = \sum_{n=0}^\infty \ell_nz^n
%    \quad\qquad \Longrightarrow \quad\qquad 
%    Q(z)=L(z/2d)
%\]
%\bigskip
%
%Nos concentramos en la función generadora exponencial
%
%\[
%    E(z) = \sum_{n=0}^\infty \ell_n\frac{z^n}{n!}
%    \quad\qquad \Longrightarrow \quad\qquad 
%    \cB E = L
%\]
%
%\pause
%
%Veamos que podemos escribir a $E_2$ en términos de $E_1$. Un bucle
%de longitud $n$ en $\bZ^2$ consta de un bucle vertical de longitud $k$
%y otro horizontal de longitud $n-k$, ambos vistos en $\bZ^1$. Esta mezcla
%de bucles no está ordenada, de manera que
%\[
%    \ell_n^{(2)}=\sum_{k=0}^n \binom{n}{k}\ell_k^{(1)}\ell_{n-k}^{(1)}
%    \quad\quad \Longrightarrow \quad\quad 
%    E_2(z) = E_1(z)^2
%    \quad\quad \Longrightarrow \quad\quad 
%    E_d(z) = E_1(z)^d
%\]
%
%
%\end{frame}
%
%
%\begin{frame}{$E_1$ y la función de Bessel modificada $I_0$}
%No hay bucles de longitud impar en $\bZ^1$. Entonces
%\[
%    \ell_n^{(1)}=\begin{cases}
%                \binom{2k}{k} & \text{si }n=2k\text{ es par} \\
%                0             & \text{si }n\text{ es impar}
%    \end{cases}
%\]
%
%
%y luego
%
%
%\[
%    E_1(z) = \sum_{k=0}^\infty \binom{2k}{k}\frac{z^{2k}}{(2k)!}
%    =\sum_{k=0}^\infty \frac{z^{2k}}{k!k!}
%    \  \Longrightarrow \  
%    I_0(2z)=\sum_{k=0}^\infty \frac{z^{2k}}{k!k!}=E_1(z),
%    \  \Longrightarrow \  
%    E_d(z) = I_0(2z)^d
%\]
%
%\bigskip
%
%Por lo tanto
%\[
%    Q_d(z) = L_d\Big(\frac{z}{2d}\Big) = \cB\Big(E_d\Big(\frac{z}{2d}\Big)\Big)=
%    \int_0^\infty I_0\Big(\frac{tz}{d}\Big)^de^{-t}\,dt.
%\]
%\end{frame}
%
%\begin{frame}{Estimando $I_0$ a través de su rep. integral}
%Tenemos
%\[
%    I_0\Big(\frac{tz}{d}\Big)=\frac{1}{\pi}\int_0^\pi e^{(tz/d)\cos\theta}
%    \,d\theta
%\]
%
%\bigskip
%
%Del principio de Laplace
%\[
%    I_0\Big(\frac{tz}{d}\Big)_0
%    \sim \mathrm{cte}\cdot (tz)^{-1/2}e^{tz/d}
%\]
%cuando $t\to\infty$, de manera que
%
%\[
%    \int_N^\infty I_0\Big(\frac{tz}{d}\Big)^d e^{-t}\,dt\sim 
%    \int_N^\infty \mathrm{cte}\cdot e^{t(z-1)}(tz)^{-d/2}\,dt
%\]
%
%La colección de funciones $t\mapsto e^{t(z-1)}(tz)^{-d/2}$
%es creciente precisamente cuando $z\to 1$ con $z$ real, por el teorema
%de convergencia monótona de Lebesgue, 
%\[
%    \lim_{\substack{z\to 1\\z\in[0,1)}}\int_N^\infty e^{t(z-1)}(tz)^{-d/2}\,dt
%    =\int_N^\infty \lim_{\substack{z\to 1\\z\in[0,1)}}e^{t(z-1)}(tz)^{-d/2}\,dt
%    =\int_N^\infty t^{-d/2}\,dt
%    .
%\]
%
%\end{frame}
%
%\begin{frame}{Análisis de $Q$}
%Es por esto que 
%\[
%    \lim_{\substack{z\to 1\\z\in[0,1]}}Q(z)=
%    \lim_{\substack{z\to 1\\z\in[0,1]}}
%    \int_N^\infty I_0\Big(\frac{tz}{d}\Big)^de^{-t}\,dt
%    \sim\mathrm{cte}\cdot\int_N^\infty t^{-d/2}\,dt 
%\]
%
%\bigskip
%
%Como la integral converge para $d\geq 3$ y diverge cuando $d=1,2$
%
%\bigskip
%\bigskip
%
%Tenemos $p=1$ cuando $d=1,2$ y $p<1$ en el caso $d\geq 3$
%
%
%\end{frame}


\end{document}
